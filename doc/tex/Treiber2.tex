% !TEX root = Projektarbeit.tex

\chapter{Das Kernelmodul - Der hardwarespezifischer Treiber} %TODO ?
Nachdem nun die Grundfunktionen eines Linux Kernelmoduls vorgestellt wurden, sollen im Folgenden die spezifischen Funktionen für die Kommunikation mit dem Waveshare e-Paper Display über die \texttt{UART}-Schnittstelle erläutert werden.

Der hardwarespezifische Treiber für das e-Paper-Display ist kein von Grunde auf neu geschriebener Treiber, sondern wird als sogenannter \glqq Stacked~Driver\grqq~gestapelt auf schon vorhandenen Low-Level-Treibern des Linux-Kernels aufgebaut. 
Um die \texttt{UART}-Kommunikation im Kernel zu ermöglichen, nutzt Treiber auf dem \texttt{serial-core} Low-Level-Treiber für serielle Kommunikation. 

\section{Automatisiertes Laden des Treibers (Hotplugging?)} %TODO titel ändern


%TODO HARDWARE Treiber und Geräte anmelden, Hotplugging 
%TODO habe noch bild dass hotplugging zeigt! für hardware erkennung usw...

\section{UART/Serial im Kernel} %TODO titel ändern

%TODO geräteinitialisierung usw..

%TODO was wäre wenn es geklappt hätte...